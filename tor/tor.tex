% In principle, this file can be redistributed and/or modified under
% the terms of the GNU Public License, version 2.
%
% However, this file is supposed to be a template to be modified
% for your own needs. For this reason, if you use this file as a
% template and not specifically distribute it as part of a another
% package/program, I grant the extra permission to freely copy and
% modify this file as you see fit and even to delete this copyright
% notice. 

\documentclass{beamer}

% There are many different themes available for Beamer. A comprehensive
% list with examples is given here:
% http://deic.uab.es/~iblanes/beamer_gallery/index_by_theme.html
% You can uncomment the themes below if you would like to use a different
% one:
%\usetheme{AnnArbor}
%\usetheme{Antibes}
%\usetheme{Bergen}
%\usetheme{Berkeley}
%\usetheme{Berlin}
%\usetheme{Boadilla}
%\usetheme{boxes}
%\usetheme{CambridgeUS}
%\usetheme{Copenhagen}
%\usetheme{Darmstadt}
%\usetheme{default}
%\usetheme{Frankfurt}
%\usetheme{Goettingen}
%\usetheme{Hannover}
%\usetheme{Ilmenau}
%\usetheme{JuanLesPins}
%\usetheme{Luebeck}
\usetheme{Madrid}
%\usetheme{Malmoe}
%\usetheme{Marburg}
%\usetheme{Montpellier}
%\usetheme{PaloAlto}
%\usetheme{Pittsburgh}
%\usetheme{Rochester}
%\usetheme{Singapore}
%\usetheme{Szeged}
%\usetheme{Warsaw}
\usepackage{textcomp}
\usepackage{graphicx}

\pgfdeclareimage[height=0.6cm]{university-logo}{iiitd-logo.png}


\title[Tor Perf \& Sec]{Tor Performance and Security}

% A subtitle is optional and this may be deleted
\subtitle{Can we be anonymous at scale ?}

\author{Muhammad F. R.~Wani}
% - Give the names in the same order as the appear in the paper.
% - Use the \inst{?} command only if the authors have different
%   affiliation.

\institute[IIIT-D] % (optional, but mostly needed)
{
  Department of Computer Science \\
  
  Indraprashta Institute \textit{of} \\
  Information Technology Delhi\\
}
% - Use the \inst command only if there are several affiliations.
% - Keep it simple, no one is interested in your street address.

\date[RM5XX -- April 2017]{Term Paper Presentation, April 2017}
% - Either use conference name or its abbreviation.
% - Not really informative to the audience, more for people (including
%   yourself) who are reading the slides online

\subject{Anonimity on Internet}
\logo{\pgfuseimage{university-logo}}

% This is only inserted into the PDF information catalog. Can be left
% out. 

% If you have a file called "university-logo-filename.xxx", where xxx
% is a graphic format that can be processed by latex or pdflatex,
% resp., then you can add a logo as follows:

% \pgfdeclareimage[height=0.5cm]{university-logo}{university-logo-filename}
% \logo{\pgfuseimage{university-logo}}

% Delete this, if you do not want the table of contents to pop up at
% the beginning of each subsection:
\AtBeginSubsection[]
{
  \begin{frame}<beamer>{Outline}
    \tableofcontents[currentsection,currentsubsection]
  \end{frame}
}

% Let's get started
\begin{document}

\begin{frame}
  \titlepage
\end{frame}

\begin{frame}{Outline}
  \tableofcontents
  % You might wish to add the option [pausesections]
\end{frame}

% Section and subsections will appear in the presentation overview
% and table of contents.
\section{Basics}

\subsection{Introduction}

\begin{frame}{Tor}{What is Tor ?}
\begin{columns}[T]
\begin{column}{0.5\textwidth}
  \begin{itemize}
  \item {The Tor network gives anonymity}
  \item {Its a type of low latency network}
  \item {Its the most commonly used service}
  \item {Comparatively easy to setup and use}
  \item {Require at-least 3 hops}
  \end{itemize}
  \end{column}
  \begin{column}{0.5\textwidth}
   \includegraphics[width=\textwidth]{tor} 
  \end{column}
  \end{columns}
\end{frame}
\begin{frame}{Tor Design}{How does it work ?}
\begin{columns}[T]

\begin{column}{0.5\textwidth}
\begin{itemize}
    \item   Over 6000+ routers called OR's 
    \item  Build a circuit and expose over socks proxy
    \item 10 min circuit idle time.
    \item Has A queuing architecture.
    \item Round robin for circuit scheduling.
    \item Randomly choose Routers.
\end{itemize}

\end{column}

\begin{column}{0.5\textwidth}
\includegraphics[width=\textwidth]{tor2}\\
Basic Tor Network\footnotemark
\footnotetext{Image Scraped from Internet}

\end{column}
\end{columns}  
\end{frame}
\subsection{Background}

% You can reveal the parts of a slide one at a time
% with the \pause command:
\begin{frame}{Design Weakness}{What's wrong ?}
  \begin{itemize}
  \item Tor is not scalable
  \item Some attacks are possible without full circuit control
  \item Transport design is an attack vector.
  \item No congestion control
\item Expensive circuit creation time.
  \end{itemize}
\end{frame}


\subsection{Motivation}
\begin{frame}{Classification of studies}{What's new and old !}
\begin{columns}[T]
  \begin{column}{0.5\textwidth}

  \begin{itemize}
      \item Traffic Management
      \item Router Selection
      \item Scalability
      \item Circuit Construction
  \end{itemize}
  \end{column}
  \begin{column}{0.5\textwidth}
  \includegraphics[scale=0.4]{research-2}
  \end{column}
\end{columns}
\end{frame}

\section{Enhancements}


\subsection{Traffic Management}
\begin{frame}{Traffic Management}{Improving scheduling of traffic}
\begin{columns}[c]
  \begin{column}{0.5\textwidth}
\begin{itemize}
    \item Incentive-based Schemes.
    \item Multi-path routing.
    \item Congestion Control.
\end{itemize}
\alert{Some important works:}
\begin{itemize}
    \item TCP over DTLS
    \item KIST
    \item UDP-OR
    \item uTOR
\end{itemize}
\end{column}
  \begin{column}{0.5\textwidth}
  \includegraphics[scale=.25]{traffic}
  \end{column}
\end{columns} 
\end{frame}

\subsection{Router Selection}
\begin{frame}{Router Selection}
    \begin{itemize}
    \item Tuneable Selection
    \item Link-Based Selection
    \item LasTOR
    \item Congestion-aware Routing
    \item Comprehensive Evalauation.
    \end{itemize}
    
\end{frame}
\subsection{Scalability}
\begin{frame}{Scalability}
\begin{columns}
  \begin{column}{0.4\textwidth}
    \begin{itemize}
        \item Peer-to-Peer Approaches
        \item PIR-Tor
    \end{itemize}
\end{column}
  \begin{column}{0.6\textwidth}
  \includegraphics[scale=1]{scalable}

\end{column}
\end{columns}
\end{frame}

\subsection{Circuit Construction}
\begin{frame}{Circuit Construction}
\begin{itemize}
   \item Improved Diffie-Hellman Based Key Agreement.
   \item Pairing Based Onion Routing.
   \item Certificateless Onion Routing.
\end{itemize}
\end{frame}

\section{Attacks \& Issues}
\subsection{Taxanomy of attacks}
\begin{frame}{Attacks}
\begin{itemize}
   \item Passive Attacks
   \begin{itemize}
       \item AS-Level Adversary
       \item Website Fingerprinting
       
   \end{itemize}
   \item Active Attacks
   \begin{itemize}
       \item End to End Confirmation Attacks
       \item Path Selection Attacks
       \item Side Channel Attacks
   \end{itemize}
\end{itemize}
\end{frame}



\subsection{Issues}

\begin{frame}{Unresolved Issues}

\begin{itemize}
\item Bot-nets
\item Blocking Resistance
\item Performance
\end{itemize}
\end{frame}

\section*{Summary}
\begin{frame}{Takeaways}

\begin{itemize}
\item Identification of key weakness of TOR.
\item Classification of research work.
\item Survey in each of the category of research.

\end{itemize}
\end{frame}

\section*{Questions}

\begin{frame}{Questions}
\centering
\includegraphics[scale=1]{questions}
\end{frame}



% All of the following is optional and typically not needed. 
\appendix
\section<presentation>*{\appendixname}
\subsection<presentation>*{For Further Reading}

\begin{frame}[allowframebreaks]
  \frametitle<presentation>{For Further Reading}
    
  \begin{thebibliography}{10}
    
  \beamertemplatebookbibitems
  % Start with overview books.

 
    
  \beamertemplatearticlebibitems
  % Followed by interesting articles. Keep the list short. 

  \bibitem{TOR}
    Tor Website.
    \newblock{\em www.torproject.org}
  \end{thebibliography}
\end{frame}

\end{document}
